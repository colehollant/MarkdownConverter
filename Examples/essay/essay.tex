\documentclass[12pt]{article}
\usepackage[margin=1in]{geometry}
\usepackage{setspace}
\usepackage[leftmargin = 1in, rightmargin = 0in, vskip = 0in]{quoting}
\usepackage{microtype}
\usepackage{amssymb, amsthm, amsmath, amsfonts}
\usepackage{ wasysym }
\usepackage{graphicx}
\usepackage{color}
\usepackage{hyperref}
\newcommand{\blockquotespacing}{\blockspaced}
\newcommand{\prose}{.25in}
\newcommand{\poetry}{0in}
\newcommand{\singlespaced}{\setstretch{1}\vspace{\baselineskip}}
\newcommand{\blockspaced}{\setstretch{1.3}\vspace{\baselineskip}}
\newcommand{\doublespaced}{}
\newenvironment{blockquote}[1][\prose]{\setlength{\parindent}{#1}\begin{quoting}\blockquotespacing}{\end{quoting}}
\begin{document}
\setstretch{2}
\raggedright
{\large Cole Hollant}\\
{\large Professor Joseph Mansky}\\
{\large LIT 2501}\\
{\LARGE Backwards Proofs and the Oedipus Complex}\\
\indent Ernest Jones provides a reading of William Shakespeare's magnum opus, \textit{Hamlet}, that suggests the prince having an Oedipus complex. He goes about this in a rather peculiar fashion of first reconstructing several other readings to curtly dismiss as incorrect, and then by offering his own interpretation. Jones thesis is that ``The actual realization of his early wish in the death of his father would then have stimulated into activity these suppressed memories, which would have produced, in the form of depression and other suffering, an obscure aftermath of his childhood's conflict'' (Jones, 93). And in the subsequent paragraph, he notes that he will support his argument with Hamlet's sentiments regarding jealousy and death, ``For the sake, however, of those who may be interested to apprehend the point of view from which this strange hypothesis seems probable I feel constrained to interpolate a few considerations on two matters that are not commonly appreciated, namely a child's feelings of jealousy and his ideas on death'' (Jones, 94).\\
\indent As much of Jones' piece consists of remarking upon the omnipresent ``inadequacy of all the solutions of the problem'' (Jones, 74), much of the article is benign to his argument; his theorizing is almost strictly found in his opening and closing remarks. One of the more interesting suggestions that Jones makes is that of \textit{Hamlet} providing commentary on the ``deeper workings of Shakspere's mind'' (Jones, 74). When Jones first mentions \textit{Hamlet} as a means of examining Shakespeare, his language--as well as his brevity--exudes uncertainty. He begins this paragraph by mentioning that ``The problem presented by the tragedy of Hamlet is one of peculiar interest in at least two respects'' (Jones, 74). This statement appears to set up for some sort of argument, and it is followed as such, but it is flimsy. He says that, ``It \textit{probably} expresses the core of Shakspere's philosophy… It \textit{may be expected}, therefore, that anything which will give us the key to the inner meaning of the play will necessarily give us the clue to much of the deeper workings of Shakspere's mind'' (Jones, 74). This, while unessential to his argument, is a telling example of his style. On one hand, Jones is quick to accept Freud's \textit{theory} of psychoanalysis as law, with language that mirrors the strength of his beliefs, ``We now know that this origin is to be found in the mental processes…It is to be expected that the knowledge so laboriously gained by the psycho-analytic method of investigation would probe of great value… inadequacy of all the solutions of the problem that have up to the present been offered'' (Jones, 73-74). He accepts psychoanalysis as a discovery/method that invalidates previous approaches, and his trust in it is marked with diction. Why is it, then, that when applying psychoanalysis to Shakespeare through \textit{Hamlet} his assurance fades so noticeably?\\
\indent After proceeding to dismiss the arguments of his fellow psychologists, he moves on to his own argument with a mix of text-based evidence, applications of psychoanalysis, and assumption of his correctness. He concludes that Hamlet's ``weariness of life'' is a result of having been, ``plunged into this abnormal state by the news of his mother's second marriage it must be because the news has awakened into activity some slumbering memory, which is so painful that it may not become conscious'' (Jones, 93). While it is clear within the play that Hamlet disapproves of Gertrude and Claudius' marriage, and that it gives him great grief, the suggestion that there's no other explanation for these circumstances to lead to Hamlet's anguish save for bringing forth repressed memories in Hamlet's unconscious mind seems heinous. This sort of argument relies minimally what is in the text, and more so on the stirrings of Hamlet's unconscious and the correctness of his Oedipus complex hypothesis. Jones begins his argument of Hamlet's Oedipus complex by noting that Gertrude is sensual and has a ``passionate fondness for her son'' (Jones, 98), and he mentions that there is bountiful evidence for each aspect that need not be explained. Why then, is the sole piece of evidence provided an observance from an untrustworthy third party in private conversation with Laertes whom he wants to murder Hamlet, and not a concrete piece of evidence? He continues providing textual analysis, suggesting that Hamlet uses Ophelia as a means of making Gertrude jealous; again, he argues this by noting it as unconscious. Then with his father's death comes with another similar argument; he relies on Hamlet having a repressed ``desire to take his father's place in his mother's affection'' (Jones, 99)--having an Oedipus complex--to say exactly that.\\
\indent Jones similarly argues that Hamlet has issues with sexuality in his relations with Ophelia because he is repressing his feelings towards Gertrude. He notes that Hamlet is jealous of Claudius because Claudius executed the will of Hamlet's unconscious. Rosencrantz, Guildenstern and Polonius are diminished to emotional outbreaks due to Hamlet's repressed sexual feelings towards Gertrude. Hamlet's inaction is explained as unwillingness to explore his unconscious mind, which again necessitates unpresent textual niceties. Jones' conclusion with commentary on Shakespeare argues that the creation of Hamlet necessitates a similar conflict within the author, and this is supported with an essentialist shrug of the Oedipus complex being present in all men. Jones' assertion that Hamlet has an Oedipus complex is inherently flawed and almost unsupportable; his argument, at large, necessitates its correctness and is backwards/circular in nature.\\
\indent While one should implore the exploration of roles, the necessity of a possibility is questionable. Jones is merely out of his lane with his piece, but it is an interesting reading that could be helpful in acting or providing a different reading. Should one want to express some Theory of Everything for Hamlet, one may keep to the text provided. Thus, Hamlet's inaction in killing Claudius may not be expressed in an unwillingness to explore his unconscious as there is no grounds for this, but it may instead be some function of seeking for assurance and proper circumstances. Hamlet's inaction may be expressed as him being uncertain as to whether or not Claudius has killed his father; this is evidenced in Hamlet's soliloquy where he hatches the plan of the reenactment play, ``The spirit that I have seen may be a dev'l, and the dev'l hath power t'assume a pleasing shape; yea, and perhaps out of my weakness and my meloncholy, as he is very potent with such spirits, abuses me to damn me. I'll have grounds more relative than this. The play's the thing wherein I'll catch the conscience of the King'' (Shakespeare and Greenblatt et al., 394). Reading in what is at face value is all that we can claim, for we cannot draw from what we do not have. When limited to what we have, Hamlet's inaction would be seen here as wanting to understand the circumstances at hand. He has seen ``the spirit'', referring to the ghost of scenes prior, and he is unsure of what it is, noting that it ``may be a dev'l'', and with that as a possibility, he may be being tricked. This trickery may be ``abusing'' him due to his ``weakness and [his] melancholy''. He then explicitly says that ``I'll have grounds more relative than this''; he notes that he is unconvinced that Claudius killed his father and wants to be sure before acting on such a notion. He announces that he will host the play to ``catch the conscience of the King'', where he will seek assurance that Claudius has committed this murder.\\
\indent Hamlet then, is convinced of Claudius' crime, and is soon after presented with an opportunity to seek his revenge when he finds Claudius praying alone, yet he again is unable to take action. ``‘A took my father grossly, full of bread, with all his crimes broad blown, as flush as May--and how his audit stands, who knows save heaven? But in our circumstance and course of thought ‘tis heavy with him. And am I then revenged to take him in the purging of his soul when he is fit and seasoned for his passage? No'' (Shakespeare and Greenblatt et al., 410). Hamlet wants to murder Claudius as Claudius murdered King Hamlet, he wants Claudius to have unresolved sin and be unable to go to heaven. The metaphor of ``full of bread'' and ``flush as May'' note that Claudius murdered Hamlet without giving Hamlet an opportunity for repentance, ``who knows save heaven?''. Hamlet then plainly says that he cannot murder Claudius when he is ``fit and seasoned for his passage'' to heaven after ``purging his soul''.\\
\indent The issue with Jones' piece is not his content, but for his methodology and supposed scope. Reading Hamlet as having an Oedipus complex is, indeed, reasonable, much of his argument is intuitive, yet still imagined and inserted into the text. I would argue that one is limited to what is in the text when searching for an explanation of the piece, although differing perspectives may still provide fun readings and interesting applications of theory, but are not essential to the text and oft cannot be seen as prerequisite.\\
\newpage
Shakespeare, William, et al. \textit{The Norton Shakespeare}. W. W. Norton \& Company, 2016.\\
Jones, Ernest. ``The Œdipus-Complex as an Explanation of Hamlet's Mystery: A Study in Motive.'' \textit{The American Journal of Psychology}, vol. 21, no. 1, 1910, pp. 72–113. JSTOR, www.jstor.org/stable/1412950.\\
\end{document}